\setupbackend[format=pdf/3]
\enabledirectives[backend.usetags=mkiv]

\mainlanguage[en]
\setuppapersize[letter]
\setupbodyfont[modern]
  
\setuptagging[state=start]
\setupinteraction[state=start]


\usemodule[amsmath]

\startdocument
  [metadata:title={Modeling and Characterization of Signals and Systems},
   metadata:subtitle={ECE 3704},
   metadata:author={Chris Wyatt},
   metadata:date={2025.6.27}]

\title{Modeling and Characterization of Signals and Systems}
C.L. Wyatt
Last Updated: \date
\blank[3*medium]


Todays lecture is a review of material from ECE 2714. We will review the modeling and characterization of continuous-time and discrete-time signals and systems.

\section{Signals}

Signals are modelled as functions $f: A \rightarrow B$ where $A$ and $B$ are sets.

Examples of some commonly encountered signals:

\startitemize
\item
  $A\in \mathbb{R}$, $B\in \mathbb{R}$: CT (Analog) signals
\item 
  $A\in \mathbb{Z}$, $B\in \mathbb{R}$: DT real-valued signals
\item 
  $A\in \mathbb{Z}$, $B\in \mathbb{C}$: DT complex-valued signals
\item
  $A\in \mathbb{R}$, $B\in \mathbb{R}^2$: two-channel CT (Analog Stero) signals
\stopitemize


To model signals we can build them up from primitive signals using transformations and combinations.

Some examples of primitive CT signals:

\startitemize
\item
$\delta(t)$, the delta function
\item
$u(t)$, the step function
\item
$e^{st}\; s\in\mathbb{C}$, the complex exponential
\stopitemize

Recall the important relations for the complex exponential. Let $s = \alpha + j\beta$, where $\alpha, \beta \in \mathbb{R}$, then

\placeformula[-]
\startformula
e^{st} = e^{(\alpha + j\beta)t} = e^{\alpha t}\, e^{j\beta t} = e^{\alpha t}\left(\cos(\beta t) + j\sin(\beta t) \right) 
\stopformula

By combining primitive signals we can create other signals. For example, the cosine function can be written as the weighted, linear combination of two complex exponentials

\placeformula[-]
\startformula
\cos(\omega t) = \frac{1}{2} e^{j\omega_0 t} + \frac{1}{2} e^{-j\omega_0 t}
\stopformula

A similar example is the Fourier Series representation of periodic CT signals, $x(t)$, that meet the Direchlet conditions. It is an infinite weighted, linear combination of complex exponentials

\placeformula[-]
\startformula
x(t) = \sum\limits_{k = -\infty}^{\infty} a_k\, e^{jk\omega_0\, t}
\stopformula

We can also think about decomposing complex signals into primitive functions rather than building them up. Recall the CT Fourier decomposition (transform) of a signal $x(t)$ is given by the improper, definite integral:

\placeformula[-]
\startformula
X(\omega) = \int\limits_{-\infty}^{\infty} x(t) e^{j\omega\, t}\; dt
\stopformula

which applies to many, but not all, signals.


Some example primitives for DT signals are:

\startitemize
\item $\delta[n]$, the discrete delta function
\item $u[n]$, the discrete unit step
\item $z^n,\; z\in\mathbb{C}$, the complex exponential
\stopitemize

Recall the DT complex exponential can be written as follows. Let $z = r\, e^{j\theta}, \, r,\theta\in\mathbb{R}$, then

\placeformula[-]
\startformula
z^n = \left( r\, e^{j\theta} \right)^n = r^n\, e^{j\theta\, n} = r^n\left(\cos(\theta\, n) + j\sin(\theta\, n) \right)  
\stopformula

Note: the type of signal, DT or CT, real-valued or complex valued, etc. can usually be inferred. To make this easier we use square brackets $[]$ to define DT signals and parenthesis $()$ to define CT signals.

\par
As before, we can construct more complex DT signals by taking products and weighted linear combimations of these primitive signals, for example:

\startitemize
\item $\left(\gamma\right)^n\; u[n],\; \gamma\in\mathbb{R}$
\item $\left(\gamma\right)^n\,\cos(\omega_0 n)\; u[n],\; \gamma,\omega_0\in\mathbb{R}$
\stopitemize

In many cases the same signals can be expressed in different forms. In the last example above

\placeformula[-]
\startformula
\left(\gamma\right)^n\,\cos(\omega_0 n)\; u[n] = \left(z\right)^n + \left(z^*\right)^n
\stopformula
where $z = \frac{\gamma}{2}\, e^{j\theta}$.

The corresponsing Fourier decomposition for a DT signal $x[n]$ is

\placeformula[-]
\startformula
X(\omega) = \sum\limits_{-\infty}^{\infty} x[n] e^{j\omega\, n}
\stopformula

a periodic function in $2\pi$, which may or may not exist for any given signal of interest.

Major classifications of signals are:

\section{Systems}

Systems either

\startitemize[n, packed]
\item produce signals
\item measure signals
\item transform signals (inputs) into other signals (outputs)
\stopitemize

\startitemize
\item CT systems transform CT signals to CT signals.
\item DT systems transform DT signals to DT signals.
\item Hybrid systems convert between CT and DT signals.

% figures here
\stopitemize

Major classifications of systems are:

\stopdocument
