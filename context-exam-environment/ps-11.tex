\startnotmode[solutions]
\setupheadertexts[Problem Set \#11]
\setupheadertexts[ECE 3704 -- Fall 2025][Due: 2025.12.8]
\stopnotmode
\startmode[solutions]
\setupheadertexts[Problem Set \#11 Solutions]
\setupheadertexts[ECE 3704 -- Fall 2025][] 
\stopmode

\startdocument
  [metadata:title={Problem Set 11},
   metadata:subtitle={ECE 3704},
   metadata:author={Chris Wyatt},
   metadata:date={\currentdate}]

\startmode[solutions]
\subject{Solutions}
\stopmode

\startnotmode[solutions]
Name: \hl[25]
\blank[28pt]

Collaborators: \hl[30]
\blank[28pt]

Honor Pledge: {\it I have neither given nor received unauthorized assistance on this assignment. I attest that
these solutions are my own.}
\blank[28pt]

Signature: \hl[25]

\subject{Instructions}

Your solutions should begin with your name and the name of any collaborators. Each problem solution should include:

\startitemize
\item the problem restated
\item the final answer boxed, and
\item the detailed derivation of your answer (text and equations).
\stopitemize

Questions requiring computations or plots should generally include the code you wrote as well as the result. Your exercise solutions must be submitted via Canvas by 11:59 pm on the due date as a single .pdf file with all pages oriented in the same direction. You should either typeset your solutions, use a PDF annotating tool, or scan work done with pencil and paper (the free Adobe Scan app works well). You can insert pages if needed but maintain the question ordering. Illegible submissions will not be considered.

\page
\stopnotmode

%=====================================================================
\startquestion Given the following circuit,
\startalignment[center] \dontleavehmode
\starttikzpicture[american voltages,scale=0.8, every node/.style={transform shape}]
\draw (-1,0) to[short,o-] (0,0);
\draw (-1,0) to[open, v=$x(t)$] (-1,-3);
\draw (0,0) to[R, l=$R_1$] (2,0);
\draw (2,0) to[C, l=$C_1$] (2,-3);
\draw (2,0) to[R, l=$R_2$] (6,0);
\draw (6,-3) to[C, l=$C_2$] (6,0);
\draw (-1,-3) to[short,o-] (6,-3);
\stoptikzpicture
\stopalignment
derive the state-space description in the form
\startformula
\dot{q} = Aq + Bx
\stopformula
where $q$ is the state vector.

\blank[big] 
\framed[width=\textwidth,location=middle, align=flushleft]{
\startmode[solutions]
\startformula
A = 
    \startmathmatrix[left={\left[}, right={\right]}, align=right]
        \NC -\left( \frac{1}{R_1 C_1} + \frac{1}{R_2 C_1}\right) \NC \frac{1}{R_2 C_1} \NR
        \NC \frac{1}{R_2 C_2} \NC -\frac{1}{R_2 C_2} \NR
    \stopmathmatrix
    \qquad
    B =
        \startmathmatrix[left={\left[}, right={\right]}, align=right]
        \NC \frac{1}{R_1 C_1} \NR
        \NC 0 \NR
    \stopmathmatrix
\stopformula
\stopmode
\startnotmode[solutions]
\blank[2cm] % adjust to leave enough space
\stopnotmode
}
\blank[big] {\bf Solution:}
\startmode[solutions]
Let $q_1$ and $q_2$ be the states corresponding to the capacitor voltages $C_1$ and $C_2$ respectively. Consider the two clockwise loop currents $i_1$ and $i_2$. Then using KVL
\startplaceformula[eq:1]
\startformula
x = R_1 i_1 + q_1
\stopformula
\stopplaceformula
and
\startplaceformula[eq:2]
\startformula
q_1 = R_2 i_2 + q_2
\stopformula
\stopplaceformula
The currents are
\startplaceformula[eq:3]
\startformula
C_1 \dot{q_1} = i_1 - i_2
\stopformula
\stopplaceformula
\startplaceformula[eq:4]
\startformula
C_2 \dot{q_2} = i_2
\stopformula
\stopplaceformula
from (\in[eq:1])
\startplaceformula[eq:5]
\startformula
i_1 = -\frac{1}{R_1} q_1 + \frac{1}{R_1} x
\stopformula
\stopplaceformula
from (\in[eq:2])
\startplaceformula[eq:6]
\startformula
i_2 = \frac{1}{R_2} q_1 - \frac{1}{R_2} q_1
\stopformula
\stopplaceformula
Substitution of (\in[eq:5]) and (\in[eq:6]) into (\in[eq:3]) and (\in[eq:4]) gives
\startformula
\startmathalignment
\NC \dot{q_1} \NC = -\left( \frac{1}{R_1 C_2} + \frac{1}{R_2 C_2}\right) q_1 + \frac{1}{R_2 C_1} q_2 + \frac{1}{R_1 C_1} x\NR
\NC \dot{q_2} \NC = \frac{1}{R_2 C_2} q_1 - \frac{1}{R_2 C_2} q_2
\stopmathalignment
\stopformula
Thus
\startformula
A = 
    \startmathmatrix[left={\left[}, right={\right]}, align=right]
        \NC -\left( \frac{1}{R_1 C_2} + \frac{1}{R_2 C_2}\right) \NC \frac{1}{R_2 C_1} \NR
        \NC \frac{1}{R_2 C_2} \NC -\frac{1}{R_2 C_2} \NR
    \stopmathmatrix
\stopformula
and
\startformula
    B =
        \startmathmatrix[left={\left[}, right={\right]}, align=right]
        \NC \frac{1}{R_1 C_1} \NR
        \NC 0 \NR
    \stopmathmatrix
\stopformula
\stopmode
\startnotmode[solutions]
\blank[10cm] % adjust to leave enough space
\stopnotmode

\stopquestion
%=====================================================================

\page

%=====================================================================
\startquestion Given the transfer function
  \startformula
  H(s) = \frac{100}{(s+5)(s^2 + 2s + 7)}\; ,
  \stopformula
  derive the state-space description in the form
  \startformula
  \dot{q} = Aq + Bx
  \stopformula
  \startformula
  y = Cq + Dx
  \stopformula
  where $q$ is the state vector, $x$ is the input, and $y$ the output.
  \blank[big] 
  \framed[width=\textwidth,location=middle, align=flushleft]{
  \startmode[solutions]
  \startformula
  A = 
    \startmathmatrix[left={\left[}, right={\right]}, align=right]
        \NC 0 \NC 1 \NC 0 \NR
        \NC 0 \NC 0 \NC 1 \NR
        \NC -35 \NC -17 \NC -7 \NR
    \stopmathmatrix
    \quad
    B =
        \startmathmatrix[left={\left[}, right={\right]}, align=right]
        \NC 0 \NR
        \NC 0 \NR
        \NC 100 \NR
\stopmathmatrix
    \quad
    C = [1 \; 0 \; 0]
    \quad   
    D = [0]
  \stopformula
  \stopmode
  \startnotmode[solutions]
  \blank[2cm] % adjust to leave enough space
  \stopnotmode
  }
  \blank[big] {\bf Solution:}
  \startmode[solutions]
 \startformula
 H(s) = \frac{Y(s)}{X(s)} = \frac{100}{(s+5)(s^2 + 2s + 7)}
 \stopformula
 or
 \startformula
 s^3 Y(s) + 7s^2 Y(s) + 17 s Y(s) + 35 Y(s) = 100 X(s)
 \stopformula
 This is equivalent to the ODE
 \startformula
  D^3 y(t) + 7D^2 y(t) + 17 D y(t) + 35 y(t) = 100 x(t)
 \stopformula
 Let $q_1 = y$, $q_2 = Dy$, and $q_3 = D^2 y$, then
 \startformula
 D^3 y = \dot{q_3} = -7 q_3 - 17 q_2 -35 q_1 + 100 x
 \stopformula
 \startformula
 \dot{q_2} = q_3
 \stopformula
  \startformula
 \dot{q_1} = q_2
 \stopformula
 Thus
  \startformula
  A = 
    \startmathmatrix[left={\left[}, right={\right]}, align=right]
        \NC 0 \NC 1 \NC 0 \NR
        \NC 0 \NC 0 \NC 1 \NR
        \NC -35 \NC -17 \NC -7 \NR
    \stopmathmatrix
    \quad
    B =
        \startmathmatrix[left={\left[}, right={\right]}, align=right]
        \NC 0 \NR
        \NC 0 \NR
        \NC 100 \NR
\stopmathmatrix
    \quad
    C = [1 \; 0 \; 0]
    \quad   
    D = [0]
  \stopformula
  \stopmode
  \startnotmode[solutions]
  \blank[10cm] % adjust to leave enough space
  \stopnotmode
  
\stopquestion
%=====================================================================

\page


%=====================================================================
\startquestion Consider the circuit from Lab 1
\startalignment[center] \dontleavehmode
\starttikzpicture[american voltages,scale=0.8, every node/.style={transform shape}]
\draw
(4,-0.5) node[op amp] (opamp1) {}
(10,-1) node[op amp] (opamp2) {}
(8,4) node[op amp, xscale=-1] (opamp3) {}
(-1,0) to[R,l=$R_G$,o-] (2,0)
(2,0) to[short] (opamp1.-)
(2,2) to[R,l=$R_1$] (5,2)
(2,4) to[R,l=$R_2$] (5,4)
(2,-3) to[R,l=$R_Q$] (2,-5)
(2,-5) node[ground] (GND1) {}
(opamp1.out) to[R,l=$R_F$] (opamp2.-)
(opamp2.+) node[ground] (GND2) {}
(2,-3) to[R,l=$R_2$] (8,-3)
(opamp3.+) node[ground] (GND3) {};
\draw (7,6) to[C,l=$C_1$] (9,6);
\draw (9,1) to[C,l=$C_2$] (11,1);
\draw (5,2)-|(opamp1.out);
\draw (2,4)--(2,2) -- (2,0);
\draw (opamp3.-) |- (9,6);
\draw (7,6) -| (opamp3.out) -- (5,4);
\draw (opamp3.-) to[R,l=$R_F$] (11,4.5);
\draw (opamp1.+) -| (2,-3);
\draw (opamp2.out) |- (8,-3);
\draw (opamp2.out) |- (11,1);
\draw (opamp2.out) |- (11,4.5);
\draw (9,1) -| (opamp2.-);
\node at (-1.5,0) {$x(t)$};
\draw (opamp2.out) to[short, -o] ++(1,0);
\node at (13,-1) {$y(t)$};

\filldraw[fill=black, radius=2pt] (2,0) circle;
\filldraw[fill=black, radius=2pt] (2,2) circle;
\filldraw[fill=black, radius=2pt] (2,-3) circle;
\filldraw[fill=black, radius=2pt] (opamp1.out) circle;
\filldraw[fill=black, radius=2pt] (opamp2.out) circle;
\filldraw[fill=black, radius=2pt] (opamp3.out) circle;
\filldraw[fill=black, radius=2pt] (opamp2.-) circle;
\filldraw[fill=black, radius=2pt] (opamp3.-) circle;
\filldraw[fill=black, radius=2pt] (11.2,1) circle;
\stoptikzpicture
\stopalignment
Derive the state-space description in the form
  \startformula
  \dot{q} = Aq + Bx
  \stopformula
  \startformula
  y = Cq + Dx
  \stopformula
  where $q$ is the state vector, $x$ is the input, and $y$ the output. Then convert that description to the transfer function $H(s)$.
  \blank[big] 
  \framed[width=\textwidth,location=middle, align=flushleft]{
  \startmode[solutions]
  \startformula
  A = \startmathmatrix[left={\left[}, right={\right]}, align=right]
        \NC 0 \NC -\frac{1}{R_F C_1} \NR
        \NC \frac{R_1}{R_2 C_2 R_F} \NC - \frac{R_1 R_2 + R_2 R_G + R_1 R_G}{R_2 R_G}\frac{R_Q}{(R_2+R_Q)C_2 R_F} \NR
    \stopmathmatrix
    \quad
    B = \startmathmatrix[left={\left[}, right={\right]}, align=right]
        \NC 0 \NR
        \NC \frac{R_1}{C_2 R_G R_F} \NR
\stopmathmatrix
  \stopformula
\startformula
C = [0 \quad 1] \qquad D = [0 \quad 0]
\stopformula
\startformula
H(s) = \frac{\frac{R_1}{C_2 R_F R_G} s}{s^2 + \frac{R_1 R_2 + R_2 R_G + R_1 R_G}{R_2 R_G}\frac{R_Q}{(R_2+R_Q)C_2 R_F}\, s + \frac{R_1}{R_2C_1C_2R_F^2}}
\stopformula
  \stopmode
  \startnotmode[solutions]
  \blank[2cm] % adjust to leave enough space
  \stopnotmode
  }
  \blank[big] {\bf Solution:}
  \startmode[solutions]
  First we choose the state variables as the capacitor voltages with the orientation indicated below.
  \startalignment[center] \dontleavehmode
\starttikzpicture[american voltages,scale=0.8, every node/.style={transform shape}]
\draw
(4,-0.5) node[op amp, label=$O_1$] (opamp1) {}
(10,-1) node[op amp, label=$O_2$] (opamp2) {}
(8,4) node[op amp, xscale=-1, label=$O_3$] (opamp3) {}
(-1,0) to[R,l=$R_G$,o-] (2,0)
(2,0) to[short] (opamp1.-)
(2,2) to[R,l=$R_1$] (5,2)
(2,4) to[R,l=$R_2$] (5,4)
(2,-3) to[R,l=$R_Q$] (2,-5)
(2,-5) node[ground] (GND1) {}
(opamp1.out) to[R,l=$R_F$] (opamp2.-)
(opamp2.+) node[ground] (GND2) {}
(2,-3) to[R,l=$R_2$] (8,-3)
(opamp3.+) node[ground] (GND3) {};
\draw (7,6) to[C,l=$C_1$, v=$q_1(t)$] (9,6);
\draw (11,1) to[C,l=$C_2$, v=$q_2(t)$] (9,1);
\draw (5,2)-|(opamp1.out);
\draw (2,4)--(2,2) -- (2,0);
\draw (opamp3.-) |- (9,6);
\draw (7,6) -| (opamp3.out) -- (5,4);
\draw (opamp3.-) to[R,l=$R_F$] (11,4.5);
\draw (opamp1.+) -| (2,-3);
\draw (opamp2.out) |- (8,-3);
\draw (opamp2.out) |- (11,1);
\draw (opamp2.out) |- (11,4.5);
\draw (9,1) -| (opamp2.-);
\node at (-1.5,0) {$x(t)$};
\draw (opamp2.out) to[short, -o] ++(1,0);
\node at (13,-1) {$y(t)$};

\node at (2,-0.5) {$v_1(t)$};
\node at (5,-1) {$v_2(t)$};

\filldraw[fill=black, radius=2pt] (2,0) circle;
\filldraw[fill=black, radius=2pt] (2,2) circle;
\filldraw[fill=black, radius=2pt] (2,-3) circle;
\filldraw[fill=black, radius=2pt] (opamp1.out) circle;
\filldraw[fill=black, radius=2pt] (opamp2.out) circle;
\filldraw[fill=black, radius=2pt] (opamp3.out) circle;
\filldraw[fill=black, radius=2pt] (opamp2.-) circle;
\filldraw[fill=black, radius=2pt] (opamp3.-) circle;
\filldraw[fill=black, radius=2pt] (11.2,1) circle;
\stoptikzpicture
\stopalignment
Next we write down the circuit equations. Consider the voltages at the nodes indicated above. Doing a KCL at node $v_1$
\startplaceformula[eq:7]\startformula
\frac{x - v_1}{R_G} + \frac{v_2 - v_1}{R_1} + \frac{q_1 - v_1}{R_2} = 0 
\stopformula\stopplaceformula
A KCL at the inverting input to $O_2$ which by the virtual ground is at a potential of zero:
\startplaceformula[eq:8]\startformula
\frac{v_2}{R_F} + C_2 \dot{q_2} = 0 
\stopformula\stopplaceformula
Similartly, a KCL at the inverting input to $O_3$ which by the virtual ground is at a potential of zero:
\startplaceformula[eq:9]\startformula
C_1 \dot{q_1} + \frac{q_2}{R_F} = 0 
\stopformula\stopplaceformula
The voltage divider at the bottom gives:
\startplaceformula[eq:10]\startformula
\frac{q_2 - v_1}{R_2} = \frac{v_1}{R_Q} 
\stopformula\stopplaceformula
We then eliminate all the variables other than the state variables and the input (i.e. $v_1$, $v_2$, and $v_3$). For example, from  (\in[eq:9]) we get the first state equation
\startformula
\dot{q_1} = -\frac{1}{R_F C_1} q_2
\stopformula

From (\in[eq:8]) we get
\startplaceformula[eq:11]\startformula
\dot{q_2} = -\frac{1}{R_F C_2} v_2
\stopformula\stopplaceformula

From (\in[eq:10]) we get
\startplaceformula[eq:12]\startformula
v_1 = \frac{R_Q}{R_2 + R_Q} q_2
\stopformula\stopplaceformula

Then substitute (\in[eq:12]) into (\in[eq:7]) and solve for $v_2$
\startplaceformula[eq:13]\startformula
v_2 = -\frac{R_1}{R_2} q_1 + \frac{R_1 R_2 + R_2 R_G + R_1 R_G}{R_2 R_G}\frac{R_Q}{R_2+R_Q} q_2 - R_1 x
\stopformula\stopplaceformula

Then substitute (\in[eq:13]) into (\in[eq:11]) to get the second state equation
\startformula
\dot{q_2} = \frac{R_1}{R_2 C_2 R_F} q_1 - \frac{R_1 R_2 + R_2 R_G + R_1 R_G}{R_2 R_G}\frac{R_Q}{(R_2+R_Q)C_2 R_F} q_2 + \frac{R_1}{C_2 R_G R_F} x
\stopformula

Thus the final form of the state equations is
  \startformula
  \dot{q} = Aq + Bx
  \stopformula

where 
 \startformula
  A = 
    \startmathmatrix[left={\left[}, right={\right]}, align=right]
        \NC 0 \NC a \NR
        \NC b \NC c \NR
    \stopmathmatrix
    =
        \startmathmatrix[left={\left[}, right={\right]}, align=right]
        \NC 0 \NC -\frac{1}{R_F C_1} \NR
        \NC \frac{R_1}{R_2 C_2 R_F} \NC - \frac{R_1 R_2 + R_2 R_G + R_1 R_G}{R_2 R_G}\frac{R_Q}{(R_2+R_Q)C_2 R_F} \NR
    \stopmathmatrix
    \quad
    B =
        \startmathmatrix[left={\left[}, right={\right]}, align=right]
        \NC 0 \NR
        \NC d \NR
\stopmathmatrix
=       \startmathmatrix[left={\left[}, right={\right]}, align=right]
        \NC 0 \NR
        \NC \frac{R_1}{C_2 R_G R_F} \NR
\stopmathmatrix
\stopformula

To find the transfer function we note that $y(t) = q_2(t)$. Thus the observation equation is simply
\startformula
y = Cq +Dx = [0 \quad 1] q + [0 \quad 0] x
\stopformula

To find the transfer function we note
\startformula
H(s) = \frac{Y(s)}{X(s)} = \frac{Q_2(S)}{X(s)} 
\stopformula

Taking the Laplace transform of the state equations and solving for $Q$ gives
\startformula
Q(s) = \left(sI - A \right)^{-1}\, B\, x
\stopformula

where
\startformula
\left(sI - A \right)^{-1} = \frac{1}{s^2 -cs -a\, b} \startmathmatrix[left={\left[}, right={\right]}, align=right]
\NC s-c \NC -a\NR
\NC -b \NC s \NR
\stopmathmatrix
\stopformula
Thus
\startformula
H(s) = \frac{ds}{s^2 -c\, s - a\, b}
\stopformula
Substituting for the constants

\startformula
H(s) = \frac{\frac{R_1}{C_2 R_G R_F} s}{s^2 + \frac{R_1 R_2 + R_2 R_G + R_1 R_G}{R_2 R_G}\frac{R_Q}{(R_2+R_Q)C_2 R_F}\, s + \frac{R_1}{R_2C_1C_2R_F^2}}
\stopformula

  \stopmode
  \startnotmode[solutions]
  \page
  {\bf 3.} (cont)
  \stopnotmode
  
\stopquestion
%=====================================================================


\page

%=====================================================================
\startquestion Given the following block diagram of a DT system
\startalignment[center] \dontleavehmode
\tikzstyle{block} = [draw, fill=gray!20, rectangle, 
      minimum height=2em, minimum width=2em]
\tikzstyle{sum} = [draw, fill=gray!20, circle, node distance=1cm]
\starttikzpicture
    \node [shape=coordinate, name=input] at (-2,0) {};
    \node [left of=input] {$x[n]$};
    \node [sum] at (2,0) (sum1) {$\Sigma$};
    \node [block] at (4,-1) (block1) {$z^{-1}$};
    \node [block] at (4,-3) (block2) {$z^{-1}$};
    \node [block] at (0,-1) (block3) {$z^{-1}$};
    \node [block] at (0,-3) (block4) {$z^{-1}$};
    \node [shape=coordinate, name=conn1] at (4,-2) {};
    \node [shape=coordinate, name=conn2] at (4,-4) {};
    \node [shape=coordinate, name=conn3] at (0, 0) {};
    \node [shape=coordinate, name=conn4] at (0,-2) {};
    \node [shape=coordinate, name=conn5] at (0,-4) {};
    \node [sum] at (2,-2) (sum2) {$\Sigma$};
    \node [sum] at (2,-4) (sum3) {$\Sigma$};
    \node [shape=coordinate, name=conn] at (4,0) {};
    \node [shape=coordinate] at (5,0) (output) {};
    \node [right of=output] {$y[n]$};
    
    \draw [->] (input) -- (sum1);
    \draw [->] (conn3) -- (block3);
    \draw [->] (block3) -- (block4);
    \draw (block4) -- (conn5);
    \draw [->] (conn4) -- node[yshift=0.3cm] {$a$} (sum2);
    \draw [->] (conn5) -- node[yshift=0.3cm] {$b$} (sum3);
    \draw (sum1) -- (conn);
    \draw [->] (conn) -- (output);
    \draw [->] (conn) -- (block1);
    \draw (block1) -- (conn1);
    \draw [->] (conn1) -- node[yshift=0.3cm] {$c$} (sum2);
    \draw [->] (conn1) -- (block2);
    \draw (block2) -- (conn2);
    \draw [->] (conn2) -- node[yshift=0.3cm] {$d$} (sum3);
    \draw [->] (sum3) -- (sum2);
    \draw [->] (sum2) -- (sum1);
\stoptikzpicture
\stopalignment
  derive the state space description in the form
  \startformula
  q[n+1] = Aq[n] + Bx[n]
  \stopformula
  \startformula
  y[n] = Cq[n] + Dx[n]
  \stopformula

  \blank[big] 
  \framed[width=\textwidth,location=middle, align=flushleft]{
  \startmode[solutions]
    \startformula
  A = 
    \startmathmatrix[left={\left[}, right={\right]}, align=right]
        \NC 0 \NC 1 \NR
        \NC d \NC c \NR
    \stopmathmatrix
    \quad
    B =
        \startmathmatrix[left={\left[}, right={\right]}, align=right]
        \NC 0 \NR
        \NC 1 \NR
\stopmathmatrix
    \quad
    C = [(b+d) \; (a+c)]
    \quad   
    D = [1]
  \stopformula
  \stopmode
  \startnotmode[solutions]
  \blank[2cm] % adjust to leave enough space
  \stopnotmode
  }
  \blank[big] {\bf Solution:}
  \startmode[solutions]
  The given block diagram is in DFI. Reversing the order of the series combination gives the direct form II.
\startalignment[center] \dontleavehmode
\tikzstyle{block} = [draw, fill=gray!20, rectangle, 
      minimum height=2em, minimum width=2em]
\tikzstyle{sum} = [draw, fill=gray!20, circle, node distance=1cm]
\starttikzpicture
    \node [shape=coordinate, name=input] at (-1,0) {};
    \node [left of=input] {$x[n]$};
    \node [sum] at (1,0) (sum1) {$\Sigma$};
    \node [block] at (3,-1) (block1) {$z^{-1}$};
    \node [block] at (3,-3) (block2) {$z^{-1}$};
    \node [sum] at (1,-2) (sum2) {$\Sigma$};
    \node [sum] at (1,-4) (sum3) {$\Sigma$};
    \node [sum] at (5, 0) (sum4) {$\Sigma$};
    \node [sum] at (5,-2) (sum5) {$\Sigma$};
    \node [sum] at (5,-4) (sum6) {$\Sigma$};
    \node [shape=coordinate, name=conn1] at (3,0) {};
    \node [shape=coordinate, name=conn2] at (3,-2) {};
    \node [shape=coordinate, name=conn3] at (3,-4) {};
    \node [shape=coordinate] at (6,0) (output) {};
    \node [right of=output] {$y[n]$};
    \draw [->] (input) -- (sum1);
    \draw [->] (sum1) -- (sum4);
    \draw [->] (conn1) -- (block1);
    \draw (block1) -- (conn2);
    \draw [->] (conn2) -- (block2);
    \draw (block2) -- (conn3);
    \draw [->] (conn2) -- node[yshift=0.3cm] {$c$} (sum2);
    \draw [->] (conn2) -- node[yshift=0.3cm] {$a$} (sum5);
    \draw [->] (conn3) -- node[yshift=0.3cm] {$d$}(sum3);
    \draw [->] (conn3) -- node[yshift=0.3cm] {$b$}(sum6);
    \draw [->] (sum3) -- (sum2);
    \draw [->] (sum2) -- (sum1);
    \draw [->] (sum6) -- (sum5);
    \draw [->] (sum5) -- (sum4);
    \draw [->] (sum4) -- (output);
    \stoptikzpicture
\stopalignment
Let $q_2 = $ output of second delay and $q_1 = $ output of first delay. Then
\startformula
\startmathalignment
\NC q_2[n+1] \NC = c q_2[n] + d q_1[n] + x \NR
\NC q_1[n+1] \NC = q_2 \NR
\stopmathalignment
\stopformula
  and
    \startformula
\startmathalignment
\NC y \NC = q_2[n+1] + a q_2[n] + b q_1[n] \NR
\NC  \NC = (a+c) q_2 + (b+d) q_1 + x \NR
\stopmathalignment
\stopformula
Thus
    \startformula
  A = 
    \startmathmatrix[left={\left[}, right={\right]}, align=right]
        \NC 0 \NC 1 \NR
        \NC d \NC c \NR
    \stopmathmatrix
    \quad
    B =
        \startmathmatrix[left={\left[}, right={\right]}, align=right]
        \NC 0 \NR
        \NC 1 \NR
\stopmathmatrix
    \quad
    C = [ (b+d) \; (a+c)]
    \quad   
    D = [1]
  \stopformula
  \stopmode
  \startnotmode[solutions]
  \blank[10cm] % adjust to leave enough space
  \stopnotmode
  
\stopquestion
%=====================================================================

\page


%=====================================================================
\startquestion  For the circuit in problem 1 where $R_1 = 100\,\text{k}\Omega$, $R_2 =50\,\text{k}\Omega$, and $C_1 = C_2 = 1\,\mu$F, use the derived state-space desciption to simulate the system when the input is the expression
  \startformula
  x(t) = \frac{t e^{-t}}{2+\cos(10 t)}u(t)
  \stopformula
  using the Runge-Kutta integrator (in C/C++ as demonstrated in class, or using the ode45 command in Matlab) for 8 seconds, starting at $t = 0$. Plot the input and all state trajectories, overlayed on the same plot, or as separate subplots. Submit your code as a zip file.

\blank[big] {\bf Solution:}
  \startmode[solutions]
  Substituting the values we get
  \startformula
  \startmathalignment
 \NC \dot{q} \NC = \startmathmatrix[left={\left[}, right={\right]}, align=right]
        \NC -30 \NC 20 \NR
        \NC 20 \NC -20 \NR
\stopmathmatrix
q + \startmathmatrix[left={\left[}, right={\right]}, align=right]
        \NC 10 \NR
        \NC 0 \NR
\stopmathmatrix
x \NR
\NC \NC= f(t,q) \NR
  \stopmathalignment
  \stopformula
  where
  \startformula
  f(t,q) = \startmathmatrix[left={\left[}, right={\right]}, align=right]
        \NC -30 q_1 + 20 q_2 + \frac{10 t e^{-t}}{2+\cos(10 t)} \NR
        \NC 20 q_1 -20 q_2 \NR
\stopmathmatrix
  \stopformula
  When simulated the resulting state space trajectories are
  \startalignment[center]
\dontleavehmode
{\externalfigure[graphics/ps11p5.pdf]}
\stopalignment
  \stopmode
  \startnotmode[solutions]
  \blank[10cm] % adjust to leave enough space
  \stopnotmode
  
\stopquestion
%=====================================================================


\stopdocument
